\usepackage{bm}

%=== To define custom Lemmas, Definitions, Theorems, also ================
%=== provides proofs with end of proof sign (\begin{proof}\end{proof}) ===
\usepackage{amsthm}
\newtheorem{def2}{Definition}
\newtheorem{conjec}{Conjecture}
\newtheorem{lemma}{Lemma}
\newtheorem{thm}{Theorem}
\newtheorem{cor}{Corollary}
\newtheorem{prop}{Proposition}
\theoremstyle{definition}
\newtheorem{remark}{Remark}
\newtheorem{case}{Case}
\newtheorem{exmp}{Example}

%=== This is for tables===========================================
\newcolumntype{L}[1]{>{\raggedright\let\newline\\\arraybackslash\hspace{0pt}}m{#1}}
\newcolumntype{C}[1]{>{\centering\let\newline\\\arraybackslash\hspace{0pt}}m{#1}}
\newcolumntype{R}[1]{>{\raggedleft\let\newline\\\arraybackslash\hspace{0pt}}m{#1}}
\newcommand{\midsepremove}{\aboverulesep = 0mm \belowrulesep = 0mm}
\midsepremove
\newcommand{\midsepdefault}{\aboverulesep = 0.605mm \belowrulesep = 0.984mm}
\midsepdefault
\newcommand\solidrule[1][1cm]{\rule[0.5ex]{#1}{.4pt}}
\newcommand\dashedrule[1][1cm]{\mbox{\solidrule[#1]\hspace{#1}\solidrule[#1]\hspace{#1}\solidrule[#1]}}

%=== symbols =====================================================
%\newcommand{\ma}[1]{\bm{ #1 }}         % matrix/vector
\newcommand{\ten}[1]{\bm{\mathcal #1}} % tensor
\newcommand{\set}[1]{{\mathcal #1}}    % set
\newcommand{\compl}{\mathbb{C}}        % complex-valued numbers
\newcommand{\real}{\mathbb{R}}         % real-valued numbers
\newcommand{\eqdef}{\stackrel{.}{=}} % definition
\DeclarePairedDelimiter{\ceil}{\lceil}{\rceil}
\DeclarePairedDelimiter{\floor}{\lfloor}{\rfloor}

%=== functional operators ======================================
\newcommand{\realof}[1]{{\rm Re}\left\{ #1 \right\}}           % real part
\newcommand{\imagof}[1]{{\rm Im}\left\{ #1 \right\}}           % imaginary part
\newcommand{\vecof}[1]{{\rm vec}\left\{ #1 \right\}}           % vec operator
\newcommand{\rect}{\rm rect}           % rect function
\newcommand{\unvecof}[2]{\mathop{\rm unvec}_{#2}\left\{ #1 \right\}}% unvec operator
\newcommand{\diagof}[1]{{\rm diag}\left\{ #1 \right\}}         % diag operator
\newcommand{\traceof}[1]{{\rm tr}\left\{ #1 \right\}}          % trace operator
\newcommand{\twonorm}[1]{\left\|#1\right\|_2}                  % two-norm
\newcommand{\fronorm}[1]{\left\|#1\right\|_{\rm F}}            % Frobenius norm
\newcommand{\honorm}[1]{\left\|#1\right\|_{\rm H}}             % higher-order norm
\newcommand{\expvof}[1]{\mathbb{E}\left\{ #1 \right\}}         % expected value
\newcommand{\expof}[1]{{\rm e}^{#1}}                           % e^{}

%=== other operators ===========================================
\newcommand{\argmax}{\mathop{{\arg \max}}}         % argmax

%=== binary operators ===========================================
\newcommand{\krp}{\diamond}  % Khatri-Rao product: A \krp B
\newcommand{\kron}{\otimes}  % Kronecker product: A \kron B

%=== superscripts ================================================
\newcommand{\degrees}{\circ}   % e.g., 90^\degrees
\newcommand{\trans}{{\rm T}}   % transpose: ^\trans
\newcommand{\herm}{{\rm H}}    % hermitian: ^\herm
\newcommand{\mtrans}{-{\rm T}} % inv.transpose: ^\mtrans
\newcommand{\mherm}{-{\rm H}}  % inv.hermitian: ^\mherm
\newcommand{\pinv}{+}          % pseudo-inverse: ^\pinv
\newcommand{\conj}{*}          % complex conjugate: ^\conj

%=== Tensor stuff ================================================
\newcommand{\tconcat}{\mbox{\Huge \textvisiblespace \hspace{-0.2ex}} \;}
\newcommand{\unf}[2]{\left[ \ten{#1} \right]_{(#2)}}
\newcommand{\unfnot}[2]{\left[ #1 \right]_{(#2)}}

%=== short-hand for colors =======================================
\usepackage{color}
\newcommand{\red}[1]{{\color{red} #1}}
\newcommand{\blue}[1]{{\color{blue} #1}}

%=== short hand for TWR quantities ===============================
\newcommand{\UT}[1]{\textsf{UT$_#1$}}
%\newcommand{\UTM}[2]{\textsf{UT$_#1^{(#2)}$}}
\newcommand{\UTM}[1]{\textsf{UT$_#1$}}
\newcommand{\RS}{{\textsf{RS}}}
\newcommand{\RSf}{\textsf{R}}
\newcommand{\RSM}[1]{\textsf{R$_#1$}}
\newcommand{\MR}{M_{\rm R}}
\newcommand{\NR}{N_{\rm R}}
\newcommand{\NP}{N_{\rm P}}
\newcommand{\NPi}[1]{N_{\rm P,#1}}
\newcommand{\PT}[1]{P_{{\rm T},#1}}
\newcommand{\PN}[1]{P_{{\rm N},#1}}
\newcommand{\PTR}{P_{{\rm T,R}}}
\newcommand{\PNR}{P_{{\rm N,R}}}
\newcommand{\PTx}{P_{{\rm T}}}
\newcommand{\PNx}{P_{{\rm N}}}
\newcommand{\eff}{{\rm (e)}}
\newcommand{\effT}{{\rm (e)^\trans}}
\newcommand{\effS}{{\rm (e)^2}}
\newcommand{\eq}{{\rm (eq)}}
\newcommand{\target}{{\rm (T)}}
\newcommand{\forw}{{\rm (f)}}
\newcommand{\forwT}{{\rm (f)^\trans}}
\newcommand{\forwH}{{\rm (f)^\herm}}
\newcommand{\backw}{{\rm (b)}}
\newcommand{\backwT}{{\rm (b)^\trans}}
\newcommand{\backwH}{{\rm (b)^\herm}}
\newcommand{\forwn}{{\rm (f,0)}}
\newcommand{\backwn}{{\rm (b,0)}}

%=== pseudo code =================================================
\newcommand{\algorithmicinitiate}{\textbf{Initiate:}}

%=== other stuff =================================================
\newcommand{\te}[1]{\mathrm{#1}} % usual text in formulas
\newcommand{\note}[1]{\red{\bfseries #1}}
\newcommand{\TODO}[1]{\par\red{\bfseries ***TODO: #1***}\\}


